\thispagestyle{plain}
\begin{center}
    \Large
    \textbf{Master Thesis}
    
    \vspace{0.4cm}
    \large
    The development and implementation of an unmanned aerial vehicle capable of external environment interaction.
    
    \vspace{0.4cm}
    \textbf{Brett Stephens}
    
    \vspace{0.9cm}
    \textbf{Abstract}
    
\end{center}

The intent of this thesis is twofold: to implement a visual inertial odometry pipeline capable of localizing a micro air vehicle in a previously unknown environment while developing a novel use case for such a pipeline. Focusing upon the first goal, the state-of-the-art visual inertial odometry landscape was quantified and a suitable pipeline was selected and catered to the needs of the Aerial Robotics Lab. The developed pipeline contains a robust method to autonomously localize and control a micro air vehicle, with the additional functionality of following a feasible trajectory. In achieving the second goal, a novel example of how this autonomous platform might be utilized is presented. Specifically, a complimentary control and trajectory algorithm is developed to allow the vehicle to interact with its environment. With the use of this algorithm, the vehicle is able to adhere to environmental objects, allowing the vehicle to gain previously unattainable proximity to such objects for a variety of potentially useful tasks. A fully functional micro air vehicle platform capable of achieving both goals is presented and discussed here, highlighting the concepts, capabilities and potential impact of such technologies. 